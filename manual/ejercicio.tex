\chapter[Ejercicios resumen.]{Ejercicio resumen para aplicar los resultados de los capítulos \ref{lineales} y \ref{crllineales}.}\label{ejercicios}

\begin{figure}[h]
\centering
\scalebox{0.65}{
\begin{tikzpicture}
    \fill[pattern=north west lines, color=gray] (0,0) rectangle ++(1,3);
    \fill[pattern=north west lines] (3,0) rectangle ++(1,3);
    \fill[pattern=north west lines] (6,0) rectangle ++(1,3);
    \fill[pattern=north west lines, color=gray] (9,0) rectangle ++(1,3);
    \fill[pattern=north west lines, color=gray] (0,-1) rectangle ++(10,1);
    \fill[pattern=north west lines, color=gray] (0,3) rectangle ++(10,1);

	\node at (2,1.5) {Hab. $1$};
	\node at (5,1.5) {Hab. $2$};
	\node at (8,1.5) {Hab. $3$};
\end{tikzpicture}
}
	\caption{Tres habitaciones separadas por muros conductores y conectadas en serie dentro de un contenedor adiabático.}
	\label{fig: adiabatico}
\end{figure}

Tenemos en la figura \ref{fig: adiabatico} un contenedor aislado adiabáticamente con $i=3$ habitaciones conectadas en serie a través de muros conductores. Cada habitación tiene una temperatura inicial $x_i(0)\in\mathbb{R}^+$ en grados Kelvin, y su variación con el tiempo sigue la siguiente ley
\begin{equation}
	\dot x_i = \sum_{j\in\mathcal{N}_i}(x_j - x_i), \label{eq: temp}
\end{equation}
donde $\mathcal{N}_i$ es el conjunto de habitaciones contiguas a la habitación $i$.

\begin{enumerate}
	\item Considera $x = \begin{bmatrix}x_1 & x_2 & x_3\end{bmatrix}^T$ y formula un modelo lineal del tipo $\dot x = Ax + Bu$ para predecir el comportamiento en el tiempo de las temperaturas de las tres habitaciones. Para ello encuentra explícitamente la solución \textbf{analítica} para $x(t)$; fíjate en (\ref{eq: solx}) y en (\ref{eq: expAJordan}). La forma de Jordan contiene información sobre los autovalores y autovectores asociados de $A$, esto te ayuda a hacer una predicción a largo plazo de los estados. Verifica el resultado simulando el sistema (\ref{eq: temp}) para una condición inicial arbitraria que cumpla $x_1(0) \neq x_2(0) \neq x_3(0)$. Sobre $B$, ten en cuenta que no hay actuación alguna en ninguno de los estados.
	\item Verifica que el \emph{principio cero} de la Termodinámica se cumple para las habitaciones cuando $t\to\infty$. Esto es, encuentra la solución analítica para $z(t) = \begin{bmatrix}z_1(t) & z_2(t)\end{bmatrix}^T$, en donde $z_1 = x_1-x_2$ y $z_2 = x_2-x_3$. Te podría ayudar el encontrar y analizar la solución del sistema $\dot z = A_z z + B_z u_z$. ¿Es coherente con el resultado del apartado anterior?
	\item Asumiendo el modelo (\ref{eq: temp}) ¿En qué habitaciones como mínimo he de medir la temperatura para poder conocer o estimar las temperaturas en todas las habitaciones para cualquier instante de tiempo $t$?
	\item Diseña un observador, que dado el mínimo número de sensores puedas reconstruir las temperaturas de todas las habitaciones. Verifica el resultado numéricamente con una simulación.
		\item Cuentas con bombas de calor/frío que se comportan de manera lineal: aumenta/disminuye la temperatura de una habitación a un grado por segundo cuando le comandas unas señal de control igual a $1$ o $-1$ respectivamente. El sistema empieza con unas temperaturas $x(t_0) = x_{t_0} = \begin{bmatrix}a & a & a\end{bmatrix}^T$, con $a \in\mathbb{R}^+$. Quisiéramos alcanzar un vector de temperaturas genérico $x(t_1) = x_{t_1}$ para $t_1 > t_0$. Entonces:
\begin{enumerate}
\item Explica cuantas bombas necesitarías como mínimo y en qué habitaciones las colocarías.
\item Una vez colocadas las bombas, diseña una señal $u(t)$ de mínima energía para llevar las temperaturas desde $x_{t_0}$ a $x_{t_1}$ en un tiempo finito $(t_1-t_0)$.
\item Diseña un controlador tal que un estado arbitrario $x_c$ sea asintóticamente estable en el sentido de Lyapunov.
\item Diseña un controlador LQR tal que un estado arbitrario $x_c$ sea asintóticamente estable pero que la energía máxima instantánea de $u$ no supere el valor de $10$.
\end{enumerate}
\end{enumerate}
