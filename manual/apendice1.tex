\appendix
\chapter[Ejemplos de codigo de simulación]{Ejemplos de código en Matlab para la simulación de sistemas dinámicos}\label{apend1}

\section[Código para el péndulo invertido]{Ejemplo de código para simular el péndulo invertido}\label{Apinv}
Partimos de las ecuaciones del movimiento en variables de estado,
\begin{align*}
\dot x_1 &= x_2\\
\dot x_2 &= \frac{g}{l}\sin x_1 - \frac{b}{ml}x_2
\end{align*}

En primer lugar creamos una función, que reproduzca las ecuaciones de estado, 
\begin{lstlisting}
function xdot = pendulo_inv(t,x,m,l,g,b)
% VARIABLES DE ENTRADA:
% t tiempo (s). Este sistema es autonomo, no depende del tiempo. 
% Evidentemente luego no se emplea el tiempo para nada.
% x Vector (columna) de estados: 
% x, vector de variables de estado, descrito como un vector columna;  
% x((1,1) angulo de pendulo con la vertical (semieje positivo) (rad);
% x(2,1) velocidad angular radianes por segundo;

% Las variables t y x deben pasarse en el orden que aparecen en la cabecera
% de la funcion: primero el tiempo y despues el vector de estados
% Este orden es necesario para que el integrador de matlab encuentre las variables

% parametros del sistema
% m Masa del pendulo (Kg)
% l longitud del pendulo (m)
% g Aceleracion de la gravedad (m/s^2)
% b coeficiente de friccion (Kg.m/s)

% VARIABLES DE SALIDA:
% xdot vector de velocidades 
% xdot(1,1) velocidad angular (derivada temporal del angulo)
% xdot(2,1) aceleracion angular (derivada temporal de la velocidad angular)

xdot(1,1) = x(2,1);
xdot(2,1) = (g*sin(x(1,1))-b*x(2,1)/m)/l;
end
\end{lstlisting}

A continuación construimos un script (puede ser un livescript) que integre el sistema representado por la función.

\begin{lstlisting}
% Definimos y damos valores los parametros del sistema
g1 = 10; % gravedad
m1 = 10; % masa
l1 = 10; % longitud
b1 = 100; % resistencia viscosa

% Definimos un intervalo de tiempo para el que queremos que el
% integrador resuelva las ecuaciones
tin = [0,15];

% Definimos unas condiciones iniciales [x_1,x_2]
x0 = [pi/3; 0]; %Siempre vector columna.

% A continuacion llamamos a uno de los integradores de matlab, 
% En este caso se trata del par encajado 
% de ordenes 4 y 5 de Dormund and Price  (ver ayuda de matlab). 
% Es el par que emplearemos habitualmente para obtener soluciones
% Atencion a la sintaxis: la funcion ode45 toma como primera variable
% un handle @ a la funcion que se desea integrar. 
% Delante de la funcion se indican entre parentesis y, por ese orden,
% las variables t y x.  
% El resto de los parametros se fijan a los valores definidos en el script.
% ademas se pasa a la funcion ode45 el intervalo de tiempo
% de integracion y las condiciones iniciales  
% pedimos a la funcion que nos devuelva los valores de tiempo (t) para 
% los que ha calculado las soluciones
% y una matriz (x) de dimensiones dim(t)  x dim(x) 
% Cada columna de esta matriz representa los valores del vector
% de estados para cada valor del tiempo contenido en el vector t
[t,x] = ode45(@(t,x)pendulo_inv(t,x,m1,l1,g1,b1),tin,x0);

% podemos representar la evolucion temporas de los estados
figure(1)
subplot(2,1,1)
plot(t,x)
title('Evolucion temporal de los estados')
xlabel('tiempo(s)')
legend('$\theta$','$\dot{\theta}$','interpreter','latex')
% y la trayectoria del sistema en el espacio de fases (Solo posible si 
% la dimension del vector de estados es 3 o menor)
subplot(2,1,2)
plot(x(:,1),x(:,2))
hold on
plot(x0(1,1),x0(2,1),'x')
title('trayectoria en el plano de fases')
xlabel('$\theta$','interpreter','latex')
ylabel('$\dot{\theta}$','interpreter','latex')
\end{lstlisting}

En las versiones modernas de Matlab es posible escribir en el mismo fichero el código correspondiente al script  y debajo  la función que representa el sistema. En versiones más antiguas, hay que escribirlos en ficheros separados y guardar ambos ficheros en el mismo directorio.

La función \texttt{ode45} tiene muchas opciones de ajuste. Veremos algunas en otros ejemplos incluidos en estos apéndices. En cualquier caso, es aconsejable leer la documentación de Matlab para conocer mejor su uso.


\section[diagramas de fase para sistemas 2D]{Ejemplo de código para obtener una representación del diagrama de fases (sistemas 2D)}\label{AVdP}

Este ejemplo se ha construido sobre el código del ejemplo anterior, para el cálculo de la dinámica del péndulo invertido.

\begin{lstlisting}
% Representacion del campo vectorial del sistema. 
% Para este calculo empezamos por obtener un conjunto de puntos
% sobre los que obtener los valores del campo. En este caso (pendulo invertido), 

% Definimos y damos valores los parametros del sistema
g1 = 10; % gravedad
m1 = 10; % masa
l1 = 10; % longitud
b1 = 100; % resistencia viscosa

% Definimos un intervalo de tiempo para el que queremos que el
% integrador resuelva las ecuaciones
tin = [0,15];

% Sabemos la posicion de los puntos de equilibrio, 
% parece interesante seleccionar una region que que contenga varios,
x1 = -3*pi/2:3*pi/18:3*pi/2;
x2 = -3:0.3:3;

% Empleamos la funcion meshgrid de matlab para construir un mallado
[x1m,x2m] = meshgrid(x1,x2);

% Calculamos las componentes de campo en los puntos del mallado
xdot1m = x2m; 
xdot2m = (g1*sin(x1m)-b1*x2m/m1)/l1;

% Empleamos quiver para representar en cada punto del mallado
% el vector asociado al campo 
figure(1)
quiver(x1m,x2m,xdot1m,xdot2m)
axis('equal')
xlabel('$\theta$','interpreter','latex')
ylabel('$\dot{\theta}$','interpreter','latex')

% Obtencion del diagrama de fases.
% Creamos un vector de valores de x1 y otro de x2 en la region de interes
%(nomalmente una region que contenga puntos de equilibrio).
% Se trata de crear un conjunto de condiciones iniciales.
x1 = -3*pi/2:3*pi/6:3*pi/2;
x2 = -3:0.6:3;
figure(2)
hold on
% Resolvemos para todas las condiciones iniciales seleccionadas
% y dibujamos los resultados. Esto puede ser muy lento
% segun el sistema y la precision que pidamos
for i = 1:length(x1)
    for j =1: length(x2)
     [t,x] = ode45(@(t,x)pendulo_inv(t,x,m1,l1,g1,b1),t,[x1(i),x2(j)]');
     plot(x1(i),x2(j),'x') %dibujo condiciones iniciales
     plot(x(:,1),x(:,2)) %dibujo trayectoria obtenida 
    end
end
xlabel('$\theta$','interpreter','latex')
ylabel('$\dot{\theta}$','interpreter','latex')

\end{lstlisting}

\chapter[Cálculo simbolico]{Ejemplo de uso de cálculo simbólico en Matlab para el estudio de la estabilidad}
Este ejemplo muestra el uso de los comandos más utilizados de matlab para apoyar el análisis de estabilidad de sistemas.

\begin{lstlisting}
%creamos unas variables simbolicas

syms x1 x2 real %real no es una variable. Sino que define x1 y x2 como
                %variable reales. 
syms s


% definimos las ecuaciones del sistema
f(1,1) =  x2
f(2,1) = -x1 -x2 + x1*x2 
%la funcion solve obtendra las soluciones de la ecuacion f(x1,x2) =(0,0) y
%por tanto los puntos de equilibrio del sistema
[X01 X02] = solve(f,x1,x2)
X0 = [X01, X02]

% Calculamos el Jacobiano
J = jacobian(f,[x1 x2])
% sustituimos en los puntos en que se anulan las ecuaciones de estado, ya que
% dichos puntos son precisamente los puntos de equilibrio. 

% uso un bucle para obtener la matriz del sistema linealizado y los
% autovalores en todos los puntos de equilibrio obtenidos. (en realidad en
% este caso solo esta el x= (0,0)
for i = 1:size(X0,1)
    Je = subs(J,{x1,x2},[X01(i),X02(i)])
    E = double(eig(Je))
    % En algunos casos, sobre todo si el sistema depende de parametros, es
    % mas facil analizar la estabilidad a partir del polinomio carateristico
    % de la matriz del sistema (Je) y emplear el metodo de Routh-Hurwitz
    % para estudiar la estabilidad a partir de los coeficientes del
    % polinomio caracteristico.
    C = det(Je - s*eye(size(Je)))
end

% funcion de liapunov para el linealizado,
% en este caso, solo hay un punto de equilibrio que es el origen. Podemos
% obtener una funcion de Lyapunov para el sistema lineal,resolviendo la
% ecuacion de lyapunov, obtenemos una funcion cuadratica V(x) = x'*P*x. 
% Despues empleamos dicha funcion como candidata de Lyapunov para estudiar
% el sistema no lineal
P =lyap(double(Je'),[1 0;0 1])
V = [x1 x2]*P*[x1;x2]
% calculamos el gradiente de la funcion de liapunov y multiplicamos por f de
% este modo obtenemos la derivada de Lee, El comando simplify, simplifica el
% resultado.
grV = gradient(V)
dotV = simplify(grV'*f)

% por supuesto, estudiar si la derivada de Lee, cumple las condiciones del
% teorema de Lyapunov o encontrar el dominio de atraccion es algo que matlab
% no puede hacer por nosotros. Hay que usar la cabeza, normal, de pensar ;)

% Empleo los metodos ya vistos en otros ejemplos para resolver el sistema
% numericamente y representar el diagrama de fases.

% ode45 va a mostrar un warning por pantalla, debido a que hay soluciones
% que divergen. Para evitarlo, anulo los warning
warning('off')
for i =-3:0.5:3
    for j = -3:0.2:3
        [t,x] = ode45(@ej11,[0 10],[i;j]);
        plot(x(:,1),x(:,2))
        hold on
        plot(i,j,'or')
    end
end
axis([-5 5 -5 5])
%r ecupero los warnings
warning('on')
% funcion f para emplear con ode45
function dotx = ej11(t,x)
dotx(1,1) = x(2,1);
dotx(2,1) = -x(1,1) - x(2,1) + x(1,1)*x(2,1);
end
\end{lstlisting}
