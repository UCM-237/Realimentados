\chapter{Diseño de controladores para sistemas en el espacio de estados}

\section{Seguidor de trayectorias deseadas}
Considera un sistema autónomo con entrada $u$ como el descrito en las ecuaciones \ref{eq:6}, \ref{eq:7} Dada una señal deseada o trayectoria $x^*(t)\in\mathbb{R}^n$, nuestro objetivo es diseñar $u(t)\in\mathbb{R}^k$ tal que $x(t) \to x^*(t)$ para $t\to\infty$.

Para ello vamos a construir la función de error
\begin{equation}
	e(t) := x(t) - x^*(t),
\end{equation}
junto con la siguiente función candidata de Lyapunov
\begin{equation}
	V(t) = \frac{1}{2}||e(t)||^2,
	\label{eq: lyaE}
\end{equation}
cuya derivada temporal satisface
\begin{equation}
	\frac{\mathrm{d}V}{\mathrm{dt}} = e^T\dot e = e^T\big(\dot x(t) - \dot x^*(t)\big) = e^Tf(x,u) - e^T \dot x^*(t).
\end{equation}
Podemos garantizar que $e(t)\to 0$ cuando $t\to\infty$ si para un conjunto compacto $\mathcal{B} := \{e : ||e|| \leq \rho, \, \rho\in\mathbb{R}_+ \}$ tenemos que $\frac{\mathrm{d}V}{\mathrm{dt}} < 0$ si $e\in\mathcal{B} \setminus \{0\}$ y $\frac{\mathrm{d}V}{\mathrm{dt}} = 0$ si $e = 0$.

\subsection{Ejemplo con un sistema cinemático de segundo orden}
Considera el siguiente sistema cinemático de segundo orden
\begin{equation}
	\ddot p(t) = u,
	\label{eq: pdyn}
\end{equation}
donde $p,u\in\mathbb{R}^l$ son las \emph{posiciones} y las \emph{aceleraciones} respectivamente en un espacio $l\in\mathbb{N}$-dimensional. Vamos a particularizar para $l=2$. Apilamos las posiciones y velocidades $p_x,p_y,\dot p_x$ y $\dot p_y$ en $x\in\mathbb{R}^4$, consideramos que podemos medir $z$, y apilamos las aceleraciones $\ddot p_x$ y $\ddot p_y$ en $u\in\mathbb{R}^2$. Por lo que las funciones $f$ y $g$ del sistema son
\begin{equation}
	\begin{cases}
		f &= \begin{bmatrix}0 & 0 & 1 & 0 \\ 0 & 0 & 0 & 1 \\
		0 & 0 & 0 & 0 \\ 0 & 0 & 0 & 0 \end{bmatrix} z + \begin{bmatrix}0 & 0  \\ 0 & 0  \\ 1 & 0 \\ 0 & 1\end{bmatrix} u \\
			g &= z.
\end{cases}
\end{equation}

Definimos como trayectorias deseadas para seguir
\begin{equation}
	p^*(t) = f(t), \quad \dot p^*(t) = f'(t),
\end{equation}
donde $f(t) \in C^2$.
Vamos a construir las siguientes señales de error
\begin{equation}
	e_1(t) = p(t) - p^*(t), \quad e_2(t) = \dot p(t) - \dot p^*(t),
\end{equation}
	para definir la siguiente función candidata de Lyapunov al estilo de (\ref{eq: lyaE}).
\begin{equation}
	V(e(t)) = \frac{1}{2}||e(t)||^2,
	\label{eq: Ve}
\end{equation}
	en donde $e(t)\in\mathbb{R}^4$ hemos apilado $e_1$ y $e_2$. La derivada temporal de ($\ref{eq: Ve}$) satisface
\begin{align}
	\frac{\mathrm{d}V}{\mathrm{dt}} &= e^T\dot e = e_1^T\dot e_1 + e_2^T\dot e_2 =  e_1^T(\dot p - f'(t)) + e_2^T(u - f''(t)) \nonumber \\
	&= e_1^Te_2 + e_2^T(u - f''(t))
\end{align}
por lo que si uno escoge
	\begin{equation}
	u = f''(t) - e_1 - e_2 = f''(t) - p(t) + f(t) -\dot p(t) + f'(t),
	\label{eq: ue}
	\end{equation}
nos lleva a
	\begin{equation}
\frac{\mathrm{d}V}{\mathrm{dt}} = -||e_2||^2 \leq 0.
	\end{equation}
	De hecho, la derivada temporal de $V(t)$ es cero sí y solo sí $e_2 = 0$. Por lo que para invocar al principio de invariance de LaSalle, debemos comprobar cual es el conjunto invariante más grande del sistema (autónomo) error $\dot e$ cuando $e_2 = 0$, esto es,
	\begin{equation}
	\begin{cases}
	\dot e_1 &= e_2 \\
	\dot e_2 &= -e_1 -e_2
	\end{cases},
	\end{equation}
	donde se puede ver que cuando $e_2 = 0$, el conjunto invariante más grande es $e_1 = e_2 = 0$. Consecuentemente, podemos concluir que $e(t) \to 0$ cuando $t\to\infty$ para cualquier $e(0)$ empezando en $\mathcal{B}$ con un $\rho$ arbitrario, esto es, tenemos convergencia global.

	\section{Diseño de ciclos límites para sistemas planos}
	Consideremos de nuevo el sistema (\ref{eq: pdyn}) para $l=2$, es decir, seguimos en 2D, o un sistema en el plano. En vez de querer seguir una trayectoria $p^*(t)$, queremos que la posición converja a un \emph{camino} cerrado que además sea un ciclo límite. El camino cerrado lo podemos definir como una curva $\phi(p^*_x, p^*_y) = 0$, es decir, todos los puntos con curva de nivel igual a cero definen \emph{el camino}. Por ejemplo, un círculo de radio $r\in\mathbb{R}_+$ 
 satisface
	\begin{equation}
	\phi(p^*) := p_x^{*2} + p_y^{*2} - r^2 = 0.
	\end{equation}
Atención a que podemos considerar $\phi(p)$ como una señal de error, i.e., $e(t) := \phi(p(t))$. Consecuentemente, podemos definir la siguiente función candidata de Lyapunov.
\begin{equation}
	V(e(t),e_v(t)) = \frac{1}{2}e(t)^2 + \frac{1}{2}||\dot p(t) - \dot p^*(t)||^2,
	\label{eq: Ve2}
\end{equation}
	cuya derivada temporal satisface (por ahora considera $\dot p^*(t) = 0$).
	\begin{equation}
		\frac{\mathrm{d}V}{\mathrm{dt}} = e\dot e + \dot p^T\ddot p = e \nabla\phi(p) \dot p + \dot p^T u,
	\end{equation}
	entonces, si uno escoge $u = -e \nabla\phi(p)^T -\dot p$, entonces tenemos que
	\begin{equation}
\frac{\mathrm{d}V}{\mathrm{dt}} = -||\dot p||^2 \leq 0,
	\end{equation}
y, similarmente a como hemos hecho en la sección anterior, por el principio de invarianza de LaSalle's concluimos que $e(t) \to 0$ cuando $t\to\infty$.

\section*{Ejercicios}
\begin{enumerate}
\item Para que $\phi(p^*)$ sea un ciclo límite, la posición no solo ha de converger a ese camino, sino que ha de evolucionar en él en el tiempo. Para ello considera $\dot p^*(t) = \begin{bmatrix}0 & 1 \\ -1 & 0\end{bmatrix}\nabla\phi(p)^T$, y concluye que ahora $\phi(p^*)$ sí es un ciclo límite.
\end{enumerate}

