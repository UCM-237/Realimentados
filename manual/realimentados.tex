\documentclass[a4paper,10pt]{book}
\usepackage[pdftex]{graphicx}
\usepackage{epstopdf}
\usepackage{subfigure}
\usepackage{amsmath,amsthm}
\usepackage{tikz}
\usepackage{circuitikz}
\usetikzlibrary{babel}
\usetikzlibrary{shapes, arrows, patterns, angles, quotes}
\textwidth= 15cm
\evensidemargin=0cm
\usepackage[spanish]{babel}
\usepackage[utf8]{inputenc}
\usepackage{textcomp}
\usepackage{amstext}
\usepackage{amsfonts}
\usepackage{amssymb}
\usepackage[hyperindex=true,breaklinks=true,colorlinks=true,linkcolor=blue]{hyperref}
\renewcommand{\tablename}{Tabla}
\renewcommand{\listtablename}{\'Indice de Tablas}

\usepackage{multirow}
\usepackage{makeidx}

%\usepackage{draftwatermark}
%\SetWatermarkText{Borrador,juan.jimenez@fis.ucm.es}
%\SetWatermarkScale{2}

% Atajos para el tikz
\tikzstyle{block} = [draw, rectangle, minimum width=6em]
\tikzstyle{sum} = [draw, fill=blue!20, circle, node distance=1cm]
\tikzstyle{input} = [coordinate]
\tikzstyle{output} = [coordinate]
\tikzstyle{pinstyle} = [pin edge={to-,thin,black}]

% Entornos para los "teoremas"
\newtheorem{definition}{Definición}
\newtheorem{algo}{Algoritmo}
\newtheorem{theorem}{Teorema}
\newtheorem{remark}{Atención}
\newtheorem{problem}{Problema}

\graphicspath{{./figuras/}}
\makeindex
\begin{document}
\title{
\begin{flushleft}
\includegraphics[width=2.5cm]{ucm2.eps}
Universidad Complutense de Madrid\\
---------------------------------------------------------------------\
\end{flushleft}
Sistemas din\'amicos y realimentaci\'on}
\author{ Juan Jim\'enez \\ H\'ector Garc\'ia de Marina}

\maketitle\
\
\vspace*{\fill}

\includegraphics[scale=1]{by-sa.eps}\\
El contenido de estos apuntes est\'e1 bajo licencia Creative Commons Atribution-ShareAlike 4.0\\
\href{http://creativecommons.org/licenses/by-sa/4.0/}{http://creativecommons.org/licenses/by-sa/4.0/}\\
\copyright Juan Jim\'enez

\bigskip
\tableofcontents
\listoffigures
\listoftables
\chapter{Introducción}
Estas notas de clase, conforman el contenido de la asignatura "Sistemas Dinámicos y Realimentación". La asignatura está especialmente orientada  a los estudiante del grado en Física y contiene tan solo dos ideas básicas:

\begin{enumerate}
\item Una parte apreciable --si no toda-- la realidad física que conocemos puede modelarse mediante el uso de lo que denominaremos sistemas dinámicos. Éstos vendrán caracterizados por un conjunto de variables de estado, que describen aspectos mensurables de la realidad y son dinámicos en el sentido de que el valor de las variables de estado evoluciona en el tiempo. Una manera habitual de describir un sistema dinámico, a partir de sus variables de estado, es mediante el uso de ecuaciones diferenciales\footnote{Para sistemas que evolucionan en tiempo continuo $t\in \mathbb{R}^+$. Para sistemas de tiempo discreto $t \in \mathbb{N}$ se emplean ecuaciones en diferencias. Por falta de tiempo, no estudiaremos sistemas de tiempo discreto en esta asignatura, aunque tienen un gran interés tanto en sí mismos como por su relación con los sistemas de tiempo continuo.}:

\begin{equation}
	\Sigma := \begin{cases}
		\dot x(t) =& f(t,x(t),u(t)) \\ y(t) =& g(t,x(t),u(t))
	\end{cases}, 
\label{eq: sigma}
\end{equation}
en donde $\dot x := \frac{\mathrm{d}}{\mathrm{dt}}(x(t))$ es la notación para la derivada total con respecto del tiempo, y $f:\mathbb{R} \times \mathbb{R}^n \times \mathbb{R}^k \to \mathbb{R}^n$ y $g: \mathbb{R} \times \mathbb{R}^n \times \mathbb{R}^k \to \mathbb{R}^m$ son funciones. Las variables de estado se agrupan en el vector $x(t)$.  El vector $u(t)$ agrupa las variables de entrada que representan magnitudes físicas externas al sistema cuyos valores influyen en la evolución de las variables de estado. Por último el vector $y(t)$ representa las salidas del sistema: magnitudes físicas, dependientes de las variables de estado y de las entradas, que pueden ser observadas (medidas). 

\item En muchos sistemas es posible manipular los valores de las variables de entrada de modo que se pueda obtener una evolución deseada de sus variables de estado y, por tanto, de su salida. El estudio de los modos de manipular las variables de entrada para conseguir la evolución deseada del sistema constituye lo que se conoce como \emph{Control de Sistemas}.

Una forma particularmente adecuada y poderosa, de manipular las variables de entrada para controlar la evolución de un sistema, es hacerlas depender de los valores que toman las variables de estado mismas: $u(t) = u(x(t))$ o de las salidas del sistema $u(t) = u(y(t))$. Esta manera de controlar un sistema se conoce como \emph{control por realimentación}.
\end{enumerate}

Podríamos decir que el estudio de los sistemas dinámicos tiene un carácter transversal ya que es aplicable a fenómenos físicos cualquier naturaleza; mecánicos, eléctricos, ópticos, etc. Además aplicando el conocido principio ''ecuaciones iguales tienen soluciones iguales'', podemos  con facilidad trasladar los resultados obtenidos en sistemas de una determinada naturaleza a otros distintos que puedan ser modelados con ecuaciones equivalentes. Así, por ejemplo, es posible establecer analogías entre sistemas mecánicos y sistemas eléctricos.

Un aspecto particularmente interesante en el estudio de los sistemas dinámicos, es el análisis de su estabilidad. En términos puramente intuitivos podríamos dar una definición preliminar de estabilidad diciendo que un sistema es estable cuando la evolución en el tiempo de sus variables de estado no diverge. A lo largo de la asignatura precisaremos este concepto, estudiaremos formas de establecer la estabilidad o no de un sistema y veremos su relación con la posibilidad de controlar un sistema mediante realimentación.

El concepto de realimentación está presente en la naturaleza. Los sistemas biológicos son capaces de reaccionar a su estado y modificarlo en beneficio propio.  Ejemplos de ello son la regulación del contenido de azúcar en sangre, de la temperatura corporal, o del equilibrio. En todos los casos, se parte del conocimiento del estado actual para actuar modificando el estado de modo que se alcance la situación deseada; a continuación, se vuelve a observar el estado resultante y se vuelve actuar, etc. Este ciclo observación-actuación-observación es un principio muy poderoso y está en la base del desarrollo científico y tecnológico. Los sistemas de control, basados en repetir dicho ciclo, son ubicuos. Están presentes en la industria, en los sistemas de comunicaciones, en el transporte y, por supuesto, en la investigación científica. Por ejemplo, el interferómetro de ondas gravitacionales  (LIGO) funciona gracias a un sofisticado sistema de control para estabilizarlo; el (gran) colisionador de hadrones (LHC) del CERN solo es una realidad gracias a los sistemas de control que permiten controlar los haces, mantener a baja temperatura sus electroimanes, etc.

La asignatura y el contenido de estos apuntes está divido en dos grandes bloques. Los temas 2, 3 y 4, son una introducción al estudio de los sistemas dinámicos no-lineales, su estabilidad y el diseño de algunos controladores básicos para este tipo de sistemas. Los capítulos 5 y 6 se centrar en el estudio de los sistemas lineales y su control por realimentación.  Se ha incluido también un apéndice con algunos ejemplos de uso de Matlab para resolver ejercicios de sistemas dinámicos.

Por último, los apuntes son para vosotros (los que estudiáis la asignatura). Están escritos en \LaTeX\ y todos los archivos fuente están colgados en el repositorio de GitHub \url{https://github.com/UCM-237/Realimentados}. Así que, si os animáis, podéis \emph{clonaros} el \emph{repo} y añadir cambiar y corregir todo lo que queráis a vuestros apuntes. Además, podéis enviarnos un \emph{pull request} cuando corrijáis errores, y así vamos mejorando los apuntes entre todos. Si no os va mucho git y/o \LaTeX\, también podéis enviar sugerencias y correcciones directamente al Campus Virtual o por correo electrónico.



\include{introduccion}
\chapter{Modelado de sistemas dinámicos}

\section{Sistemas en el espacio de estados}
Nos vamos a centrar en sistemas que puedan ser descritos por características cuantificables. A estas características las vamos a llamar {\bf estados}, como por ejemplo, una temperatura, una velocidad, o un voltaje. Si estos estados dependen del tiempo, entonces, llamamos {\bf señal} a la sucesión de valores de los estados en el tiempo. Uno podría interaccionar con el sistema a través de una {\bf entrada} cuantificable, y a su vez medir información del sistema a través de una {\bf salida} cuantificable.

Vamos a definir $x(t)\in\mathbb{R}^n$, $y(t)\in\mathbb{R}^m$ y $u(t)\in\mathbb{R}^k$ como el vector apilado de estados, la salida, y la entrada a un sistema $\Sigma$ respectivamente. En particular, son señales, e.g., $x :[0,\infty) \to \mathbb{R}^n$.

El sistema $\Sigma$ es un modelo que predice el valor de los estados y la salida a lo largo del tiempo. Esta predicción incorpora la interacción de la entrada con los estados y la salida. En particular, vamos a emplear ecuaciones diferenciales como herramienta para predecir la evolución en el tiempo de los estados del sistema $\Sigma$ como se muestra a continuación
\begin{equation}
	\Sigma := \begin{cases}
		\dot x(t) =& f(t,x(t),u(t)) \\ y(t) =& g(t,x(t),u(t))
	\end{cases}, 
\label{eq: sigma}
\end{equation}
en donde $\dot x := \frac{\mathrm{d}}{\mathrm{dt}}(x(t))$ es la notación para la derivada total con respecto del tiempo, y $f: \mathbb{R}^n \times \mathbb{R}^k \to \mathbb{R}^n$ y $g: \mathbb{R}^n \times \mathbb{R}^k \to \mathbb{R}^m$ son funciones.

Podemos representar el sistema $\Sigma$ como un bloque con puertos de entrada y de salida como se muestra en la figura \ref{fig: sigma}.

\begin{figure}[!h]
\centering
\begin{tikzpicture}[auto, node distance=2cm,>=latex']
	\node [input, name=input] {};
	\node [block, right of=input] (system) {$\Sigma$};
	\node [output, right of=system] (output) {};
	\draw [draw,->] (input) -- node {$u(t)$} (system);
	\draw [->] (system) -- node [name=y] {$y(t)$}(output);
\end{tikzpicture}
	\caption{Diagrama de bloque entrada/salida del sistema $\Sigma$.}
	\label{fig: sigma}
\end{figure}

Dado que el sistema viene descrito por las variables de estado,  con frecuencia es posible prescindir de la salida y centrar el estudio en la primera de las dos ecuaciones anteriores.

 En ocasiones podemos emplear para describir un sistema ecuaciones de estado en las que no aparece explícitamente la entrada,
\begin{equation}
\dot x = f(t,x)
\end{equation}\label{eq: for}
En este caso se, se habla de una ecuación de estado no forzada. Esto no quiere decir necesariamente que la entrada al sistema sea cero. Puede aparecer representada directamente como una función del tiempo, $u=\rho(t)$,  como una función (\emph{realimentación}) de los estados $u = \rho(x)$ o como una función de ambos $u=\rho(t,x)$.
Un caso especialmente interesante de (\ref{eq: for}) se obtiene cuando $f$ no depende explícitamente del tiempo,
\begin{equation}
\dot x = f(x)
\end{equation}
Se trata entonces de un sistema \emph{autónomo} o \emph{invariante en el tiempo}. Es decir, el comportamiento del sistema es invariante bajo traslaciones del origen de tiempos; si cambiamos la variable tiempo de $t$ a $t-a$, la parte derecha de la ecuación de estado no experimenta cambio.

Para un sistema dinámico, descrito mediante variable de estados, un punto de equilibrio $x=x^*$  tiene la propiedad de que si el sistema se encuentra en $x*$ en un instante de tiempo $t_0$, permanecerá en ese  mismo punto para todo instante de tiempo posterior $t>t_0$. En el caso de un sistema autónomo, los puntos de equilibrio son las raíces reales de la ecuación,
\begin{equation}
f(x)=0
\end{equation} 

Los puntos de equilibrio de juegan un papel muy importante en el estudio de los sistemas dinámicos, como tendremos ocasión de ver más adelante.

Un caso particular de sistema dinámicos los constituyen los sistemas lineales. Para ellos la ecuación \ref{eq: sigma} toma la forma,
\begin{equation}
	\Sigma := \begin{cases}
		\dot x(t) =& A(t)x(t)+B(t)u(t) \\ y(t) =& C(t)x(t)+D(t)u(t)
	\end{cases}, 
\label{eq: sigmaL}
\end{equation}

En el capítulo 5 se estudiarán en detalle los sistemas lineales, cuyas propiedades permiten emplear herramientas de análisis precisas y potentes para su estudio y caracterización. En el caso del resto de los sistemas, es decir todos aquellos que son no-lineales, el análisis resulta más complejo.

Una primera aproximación al estudio de los sistemas no-lineales es linealizarlos en torno a punto de trabajo y estudiar el sistema lineal resultante. Esta aproximación es muy valiosa, como  veemos en la sección (\ref{sec: linear}), sin embargo, solo es válida en las proximidades del punto de trabajo, por tanto no es posible predecir a partir de ella el comportamiento del sistema cuando nos alejamos de dicho punto. Por otro lado, los sistemas no-lineales presentan una dinámica mucho mas rica, con fenómenos que no se dan en los sistemas lineales y, por tanto no pueden describirse con un modelo linealizado del sistema no-lineal. Algunos de éstos fenómenos son:

\begin{itemize}


\item \textbf{Múltiples puntos de equilibrio aislados.} Para un sistema lineal solo es posible encontrar un punto de equilibrio.  Un sistema lineal es estable cuando, partiendo de valores iniciales arbitrarios, sus variables de estado se aproximan al punto de equilibrio con el tiempo. 

\begin{equation}
\lim_{ t \to \infty} x \to x^*
\end{equation}

Un sistema no lineal puede presentar más de un punto de equilibrio y aproximarse a uno u otro dependiendo de las condiciones iniciales.

\item \textbf{Tiempo de escape finito (\emph{Finite escape time}).} Un sistema lineal es inestable cuando, partiendo de valores arbitrarios, sus variables de estado divergen --se van a infinito-- con el tiempo.
\begin{equation}
\lim_{ t \to \infty} x \to \infty
\end{equation}
Es posible encontrar sistemas no-lineales inestables que se comportan de modo análogo a los lineal pero además, algunos sistemas lineales divergen, se van a infinito, en un tiempo finito. 
\begin{equation}
\lim_{ t \to t_e} x \to \infty
\end{equation}

\item \textbf{Cíclos límite.} Un sistema lineal se dice que es marginalmente estable, cuando sus variables de estado oscilan periódicamente. Sin embargo, como veremos más adelante, las condiciones para que se de una oscilación man en un sistema lineal, constituyen una condición no robusta, por lo que es casi imposible que se de en presencia en condiciones reales. Además la amplitud de las oscilaciones depende de las condiciones iniciales del sistema.

Hay sistemas no lineales que pueden oscilar de modo estable con una amplitud y frecuencia independiente de las condiciones iniciales. A este tipo de oscilaciones se les conoce con el nombre de ciclos límite. 
\end{itemize}

 Resulta por tanto necesario, buscar herramientas específicas para el análisis de los sistemas no-lineales.
\section{Ejemplos} \label{sec: ejem}
\subsection{Péndulo invertido}

Vamos a derivar las funciones $f$ y $g$ en (\ref{eq: sigma}) para el sistema del péndulo invertido.

Primero, vamos a hallar la ecuación diferencial que describe la dinámica de una masa $m\in\mathbb{R}_+$ en el extremo de un péndulo de longitud $l\in\mathbb{R}_+$ tal y como se muestra en la figura \ref{fig: invpen}. Vamos a considerar que podemos interactuar con el sistema por medio de un torque $T\in\mathbb{R}$ en el otro extremo del péndulo, que la masa sufre un rozamiento proporcional $b\in\mathbb{R}_+$ a su celeridad, y que podemos medir el ángulo $\theta\in\mathbb{R}$ que forma el péndulo con la vertical.

\begin{figure}[!h]
\centering
\begin{tikzpicture}
    \coordinate (origo) at (0,0);
    \coordinate (pivot) at (1,5);

    % draw axes
	\fill[black] (origo) circle (0.05) ++ (-0.25,0.25) node (T) [black] {$T$} ++(0.75,0.75) node (l) [above] {$l$};
    \draw[thick,gray,->] (origo) -- ++(4,0) node[black,right] {$x$};
    \draw[thick,gray,->] (origo) -- ++(0,4) node (mary) [black,above] {$y$};

    % draw roof
    \fill[pattern = north east lines] ($ (origo) + (-1,0) $) rectangle ($ (origo) + (1,-0.5) $);
    \draw[thick] ($ (origo) + (-1,0) $) -- ($ (origo) + (1,0) $);

    \draw[thick] (origo) -- ++(-300:3) coordinate (bob) ++(0.5,0) node (hola) [black] {$m$};
    \fill (bob) circle (0.2);
    \draw[->] (bob) -- ++(0,-1) node (mg) [right] {$mg$};
    \draw[->] (bob) -- ++(-0.5,0.5) node (fr) [above] {$b\dot\theta$};

    \pic [draw, <-, "$\theta$", angle eccentricity=1.5] {angle = bob--origo--mary};
\end{tikzpicture}
\caption{Péndulo Invertido}
\label{fig: invpen}
\end{figure}

Define $g = 9.8$ y $I\in\mathbb{R}_+$ como la aceleración gravitatoria y el momento de inercia del péndulo respectivamente. Es sencillo comprobar que $I = ml^2$, y explotaremos que $I \ddot\theta = \text{suma de torques}$. De hecho, tenemos que considerar tres torques: 1. Torque $T$ ejercido por nosotros en la base del péndulo; 2. Torque $-bl\dot\theta$ ejercido por la fricción en la masa; 3. Torque $mgl \sin\theta$ ejercido por la atracción gravitatoria. Por lo que la ecuación diferencial que modela el comportamiento del péndulo invertido es
\begin{equation}
\ddot\theta = \frac{1}{ml^2}\left(mgl\sin{\theta}-bl\dot\theta + T\right).
\label{eq: dyn}
\end{equation}

Parece razonable escoger $\theta$ como uno de los estados para construir $x(t)$ en (\ref{eq: sigma}). De hecho, la ecuación (\ref{eq: dyn}) es de segundo orden en $\theta$, por lo que es conveniente escoger $\dot\theta$ como un estado también. Por lo tanto, definamos nuestro vector de estados como
\begin{equation}
x := \begin{bmatrix}\theta \\ \dot\theta \end{bmatrix},
\end{equation}
y como el torque $T$ es como interactuamos con el sistema, escogemos como entrada $u(t) = T(t)$.

Ahora estamos listos para construir las funciones $f$ and $g$ en (\ref{eq: sigma}) para el péndulo invertido. Atención a que $f$ y $g$ solo toma como argumentos los vectores de estados $x$ y de entradas $u$. En el lado izquierdo de (\ref{eq: sigma}) tenemos la derivada temporal de $x(t)$, por lo que

\begin{equation}
	\frac{\mathrm{d}}{\mathrm{dt}}\left(\begin{bmatrix}\theta \\ \dot\theta \end{bmatrix}\right) = f(x(t), u(t)) = \begin{bmatrix}f_1(x(t), u(t)) \\ f_2(x(t), u(t))\end{bmatrix}, \label{eq: fn}
\end{equation}
donde automáticamente obtenemos que $f_1 = \dot\theta$. Fijarse que la primera fila de $f$ en (\ref{eq: fn}), a la izquierda tenemos que $\frac{\mathrm{d}}{\mathrm{dt}}\theta$, y a la derecha tenemos que $f_1 = \dot\theta$ porque $\dot\theta$ es un estado o elemento de $x$. Desafortunadamente, no podemos decir que $f_2 = \ddot\theta(t)$ porque $ \ddot\theta$ no es un estado o elemento de $x$. No obstante, tenemos que la segunda fila $f_2$ viene dada por la ecuación (\ref{eq: dyn}). Por lo que podemos escribir $f$ como
\begin{equation}
	\frac{\mathrm{d}}{\mathrm{dt}}\left(\begin{bmatrix}\theta \\ \dot\theta \end{bmatrix}\right) =  f(x(t), u(t)) = \begin{bmatrix} \dot\theta \\ \frac{1}{ml^2}\left(mgl\sin{\theta}-bl\dot\theta + T\right) \end{bmatrix}. \label{eq: f}
\end{equation}

El cálculo de $g$ es más sencillo en este caso. Hemos establecido al comienzo que solo podemos medir el ángulo $\theta$. Por lo que $y(t) = \theta(t)$, i.e.,
\begin{equation}
g(x(t),u(t)) =  \theta(t).
	\label{eq: g}
\end{equation}

%A Python simulation of this dynamics can be found at \url{https://github.com/noether/aut_course}.

\subsection{El oscilador de Van der Pol}
Se trata de un oscilador, propuesto por primera vez por Balthasar Van der Pol, cuando trabajaba en Philips, para explicar las oscilaciones observadas en tubos de vacío. Podemos obtener la ecuación del oscilador, empleando el circuito de la figura \ref{fig:vdp}.
\begin{figure}
\centering
\begin{circuitikz}[american, scale = 0.6]\draw
(0,-4)to[short]
(4,-4)to[short,,i^<= $i_C$]
(5,-4)to[short,i=$i_N$]
(6,-4)to[short](10,-4)
(0,-7.5)to[C = C](0,-4)
(5,-4) to[L = L, i>^= $i_L$,*-* ](5,-7.5)
(10,-4) to[generic=NL,  i= $i_N \equiv h(v)$](10,-7.5) 
(0,-7.5)to[short](10,-7.5)
;
\end{circuitikz}
\caption{Circuito eléctrico no lineal}
\label{fig:vdp}
\end{figure}

Donde el elemento no lineal NL, presenta una relación entre voltaje e intensidad caracterizada por la función $h(v)$.

El voltaje $v$ en los tres componentes del circuito debe ser igual; además,
\begin{align}
v = L \frac{di_L}{dt}\\
i_C = C\frac{dv}{dt}
\end{align}
Si aplicamos la primera ley de Kirchohff al nodo superior del circuito,
\begin{align}
i_C+i_L+i_N = 0\\
C\frac{dv}{dt}+\frac{1}{L}\int_{-\infty}^{t}v(s)ds +h(v)=0\label{eq:vdp}
\end{align}
Si derivamos (\ref{eq:vdp}) con respecto al tiempo, dividimos por $C$ y reordenamos,
\begin{equation}\label{eq:vdp2}
\frac{d^2v}{dt^2}  + \frac{1}{C}\frac{dh(v)}{dv}\frac{dv}{dt} + \frac{1}{LC}\cdot v= 0
\end{equation}
Se trata de un caso particular de la ecuación de Liénard,
\begin{equation}
\ddot{v} +f(v)\dot{v}+g(v) = 0
\end{equation}
Si definimos ahora, $h(v)$,
\begin{align}
h(v) = m(\frac{1}{3}v^3-1)\\
\frac{dh}{dv} = m(v^2-1)
\end{align}
y sustituyendo en (\ref{eq:vdp2}),
\begin{equation}
\ddot{v} +m\frac{1}{C}(v^2-1)\dot{v}+\frac{1}{LC}v = 0
\end{equation}

Podemos representarla finalmente en variables de estado, tomando $x_1=v$ y $x_2=\dot{v}$,
\begin{align}
\dot{x}_1 &= x_2\\
\dot{x}_2 &= -\frac{1}{LC}x_1 - m\frac{1}{C}(x_1^2-1)x_2 \label{eq:vdpst2}
\end{align}
Podemos ahora hacer un primer análisis cualitativo. El primer término a la derecha del igual en  la ecuación (\ref{eq:vdpst2}) representa una fuerza recuperadora proporcional al desplazamiento, el segundo término, crecerá con la velocidad para $x_1 < 1$, alejando así al sistema del origen  y representará un término disipativo para $x_1 > 1$ acercándolo por tanto de nuevo al origen. Es por tanto esperable, que se alcance
algún tipo de situación de equilibrio. Más adelante definiremos esta situación rigurosamente como un ciclo límite. 
\subsection{Un vehículo de cuatro ruedas. }
La figura \ref{fig:vehir} muestra un esquema de un vehículo terrestre de cuatro ruedas, visto desde arriba. Si consideramos que se mueve en el plano $x,y$, y que su velocidad instántanea $\vec{V}$ está siempre orientada en la dirección de avance del vehículo $\psi$ --asumimos que no derrapa, ni se mueve lateralmente--, podemos entonces definir la velocidad, en el sistema de referencia $x,y$ como,
\begin{align}
\dot{x} = V_x(t) = V\cos(\psi(t))\\
\dot{y} = V_y(t) = V\sin(\psi(t)).
\end{align}

\begin{figure}
\centering

\begin{tikzpicture}
\draw(0,0)[rotate around={30:(0,0)},very thick,draw = blue!60!black!40]rectangle(2,1);
\draw(-0.3,-0.2)[rotate around={30:(0,0)},very thick,draw = blue!60!black!40]rectangle(0.3,0.0);
\draw(-0.3,1)[rotate around={30:(0,0)},very thick,draw = blue!60!black!40]rectangle(0.3,1.2);
\draw({2*cos(30)-0.3},{2*sin(30)-0.1})[rotate around={60:({2*cos(30)},{2*sin(30)})},very thick,draw = blue!60!black!40]rectangle({2*cos(30)+0.3},{2*sin(30)+0.1});
\draw({2*cos(30)-sin(30)-0.3},{2*sin(30)+cos(30)-0.1})[rotate around={60:({2*cos(30)-sin(30)},{2*sin(30)+cos(30)})},very thick,draw = blue!60!black!40]rectangle({2*cos(30)-sin(30)+0.3},{2*sin(30)+cos(30)+0.1});
\draw[blue]({cos(30)-0.5*sin(30)},{sin(30)+0.5*cos(30)})--node[above]{$\vec{V}$}({4*cos(30)-0.5*sin(30)},{4*sin(30)+0.5*cos(30)})[-latex];
\draw[red]({cos(30)-0.5*sin(30)},{sin(30)+0.5*cos(30)})--node[above]{$\vec{v'}$}({cos(30)-0.5*sin(30)+3*cos(60},{sin(30)+0.5*cos(30)+3*sin(60})[-latex];

\draw({cos(30)-0.5*sin(30)},{sin(30)+0.5*cos(30)})--({3+cos(30)-0.5*sin(30)},{sin(30)+0.5*cos(30)})[-latex]node[anchor=north]{x};
\draw({cos(30)-0.5*sin(30)},{sin(30)+0.5*cos(30)})--({cos(30)-0.5*sin(30)},{3+sin(30)+0.5*cos(30)})[-latex]node[anchor=east]{y};

\draw[blue]({2+cos(30)-0.5*sin(30)},{sin(30)+0.5*cos(30)})arc(0:30:2);
\draw[red]({3*cos(30)-0.5*sin(30)},{3*sin(30)+0.5*cos(30)})arc(30:60:2);
\draw[blue]({3.2*cos(30)},{3.2*sin(30)}) node[]{$\psi$};
\draw[red]({3.4*cos(50)},{3.3*sin(50)}) node[]{$\phi$};
\draw[latex-latex](0.25,-0.3)--node[below]{l}({2*cos(30)+0.25},{2*sin(30)-0.3});
\end{tikzpicture}
\caption{Esquema de un vehículo terrestre de 4 ruedas}
\label{fig:vehir}
\end{figure}

Además el vehículo girará, siempre que las ruedas delanteras no estén alineadas con las ruedas traseras, cambiando así su dirección de avance . Podemos relacionar la velocidad de giro del vehículo $\dot{\psi}$ con el ángulo de orientación de las ruedas delanteras $\phi$, y la velocidad a la que avanzan $\vec{v}'$.  podemos obtener las componentes de dicha velocidad en ejes cuerpo (paralela y perpendicular a la dirección de avance del vehículo),

\begin{align}
v_{P} = v'\cos(\phi(t))\\
v_{T} = v'\sin(\phi(t))
\end{align} 

Pero la rueda esta unida al vehículo así que su velocidad en la direccíon de avance debe ser la misma que la del vehículo: $v_P \equiv V$.  A partir de esta relación podemos obtener la velocidad tangencial de las ruedas como,
\begin{equation}
v_T = V\frac{\sin(\phi)}{\cos(\phi)} = V\tan(\phi)
\end{equation}

Si tomamos como centro de giro del vehículo el centro de su eje trasero, y la batalla (distancia entre ejes) es $l$, obtenemos una expresión para su velocidad de giro,
\begin{equation}
\dot{\psi} = \frac{V}{l}\tan(\phi)
\end{equation}

En resumen, podemos describir el sistema mediante tres ecuaciones de estado $x_1 \equiv x, x_2\equiv	y, x_3 \equiv \psi$:

%\begin{align}
%\dot{x}_1 =  V\cos(x_3)\\  
%\dot{x}_2 =  V\sin(x_3)\\
%\dot{x}_3 = \frac{V}{l}\tan(\phi)
%\end{align}

\begin{equation}
	\begin{cases}
		\dot x_1 &=  V\cos(x_3)\\  
		\dot x_2 &=  V\sin(x_3)\\
		\dot x_3 &= \frac{V}{l}\tan(\phi)
	\end{cases}
\end{equation}

Si consideramos $V=cte$, la única entrada al sistema sería el ángulo de giro de las ruedas $u(t) = \phi(t)$, controlando su valor, podemos hacer girar al vehículo en la dirección deseada. 

\section{Simulación o soluciones numéricas del sistema $\Sigma$}
Dado un punto inicial $x(0)$, podemos predecir o calcular numéricamente $x(t)$. El método de \emph{integración de Euler} es un método numérico sencillo que puede darnos información sobre la evolución temporal de los estados y salidas de $\Sigma$. El siguiente algoritmo describe la integración numérica por Euler:
\begin{algo}
	\begin{enumerate}
		\item Define el paso de tiempo $\Delta T$
		\item Define $x = x(0)$
		\item Define $y = g(x,u)$
		\item Registra $x$ and $y$, para poder procesarlos después si fuera necesario
		\item Define $t = 0$
		\item Define un tiempo final $t^*$
		\item Mientras $t \leq t^*$ entonces:
			\begin{enumerate}
				\item $x_{\text{nuevo}} = x_{\text{viejo}} + f(x_{\text{viejo}},u)\Delta T$
				\item $y_{\text{nuevo}} = g(x_{\text{nuevo}},u)$
				\item Representa $x$ gráficamente
				\item Registra $x$ e $y$, para poder procesarlos más adelante si fuera necesario
				\item $t = t + \Delta t$
			\end{enumerate}
		\item Representa $x$ e $y$ a lo largo de $t$
	\end{enumerate}
\end{algo}

Este algoritmo rinde bien cuando $\Delta T$ es suficientemente pequeño en función de como de rápido varíe $f$ en el tiempo. Por ahora hemos considerado $u=0$, es decir, no hay control o interacción alguna con el sistema.

\section{Sistemas autónomos de segundo orden}
Los sistema autónomos de segundo orden, son especialmente atractivos para el estudio de los fenómenos no lineales porque sus soluciones se pueden representar fácilmente en el llamado plano e fases.
En general, podemos definir un sistema autónomo de segundo orden a partir de dos ecuaciones escalares, una para la derivada temporal de cada estado,
\begin{align}
\dot x_1 &= f_1(x_1,x_2) \label{eq: sysa1}\\
\dot x_2 &= f_2(x_1,x_2) \label{eq: sysa2}
\end{align}

Supongamos que conocemos una solución al sistema $x(t) = [x_1(t),x_2(t)]^T$, que pasa por el punto $x(0) = x_0$. La solución describe una curva sobre el plano $x_1-x_2$ que pasa por el punto $x_0$. Esta curva recibe el nombre de trayectoria u órbita, desde $x_0$, del sistema. El plano $x_1-x_2$ recibe el nombre de plano de fases. La ecuaciones (\ref{eq: sysa1}) - (\ref{eq: sysa2}) representa un vector tangente $\dot x(t) = [\dot x_1(t),\dot x_2(t)]$ a la trayectoria en el punto $x(t)$ .

Además podemos considerar $f(x)=[f_1(x),f_2(x)]$ como un campo vectorial definido sobre el plano de fases., es decir asignamos a cada punto $x$ en el espacio de fases el vector $f(x)$. Es posible obtener una representación aproximada del campo vectorial asociado a un sistema si definimos una región de interés del plano de fase, por ejemplo, una región que contenga sus puntos de equilibrio. Seleccionamos un conjunto de puntos $x$ en posiciones equiespaciadas dentro de la región de interés, y representamos en cada punto $x$ el campo asociado $f(x)$ mediante una flecha que apunte en la dirección de $f(x)$ y cuya longitud sea proporcional al módulo de $f(x)$.

El conjunto de todas las trayectorias en el plano de fases de un sistema se denomina su diagrama de fases. de modo análogo al caso de campo vectorial, podemos obtener una representación aproximada del diagrama de fases obteniendo las trayectorias del sistemas para un conjunto suficientemente grande de condiciones iniciales en una región de interés y dibujándolas en el plano de fases.

Las representaciones en el plano de fases, no ofrecen un camino sencillo para observar algunos de los fenómenos propios de los sistemas no lineales. A continuación, vamos a emplearlas con algunos de los ejemplos de la sección \ref{sec: ejem}

\subsection{Diagrama de fases para el péndulo invertido}
Podemos obtener un sistema autónomo para el péndulo invertido, eliminando la tensión $T$ de las ecuaciones \ref{eq: f},

\begin{align}
\dot x_1 = &x_2\\
\dot x_2 = &\frac{g}{l}\sin x_1 - \frac{b}{l}x_2
\end{align}
donde $x_1 = \theta$ y $x_2 = \dot \theta$,. Una primera característica del sistema la obtenemos calculando los puntos de equilibrio, $x*=[n\pi,0]^T, n \in \mathbb{Z}$. Tenemos un conjunto infinito de posibles puntos de equilibrio, correspondiente a las posiciones verticales del péndulo.

La figura \ref{fig:trpen} muestra una trayectoria del péndulo invertido para una realización particular; $l=g=b$ y una condición inicial $x_0=[\pi/3,0]$.






\chapter{Comportamiento dinámico y estabilidad}
El concepto de estabilidad y su análisis constituye unos de los aspectos claves para el estudio de los sistemas dinámicos. La estabilidad de un sistema esta estrechamente relacionada con su comportamiento dinámico y puede definirse de diversas maneras. En este capítulo nos centraremos en el análisis de los llamados puntos de equilibrio de un sistema y los estudiaremos de acuerdo con el concepto de estabilidad de Lyapunov. Además, incidiremos en otros aspectos de la estabilidad de los sistemas tales como la existencia de ciclos límite o el movimiento nominal. Empecemos pues por el estudio de los puntos de equilibrio.

\section{Sistemas autónomos}

\subsection{Puntos de equilibrio} 
\begin{definition}[Punto de equlibrio] Dados un sistema dinámico general definido por las ecuaciones,
\begin{align}
\mathbf{\dot{x}} = \mathbf{f}(\mathbf{x},\mathbf{u})\\
\mathbf{y} = \mathbf{g}(\mathbf{x},\mathbf{u})
\end{align}
donde $\mathbf{x} \in \mathbb{R}^n$, $\mathbf{u} \in \mathbb{R}^m$ e $\mathbf{y} \in \mathbb{R}^l$. Se definen como puntos de equilibrio o puntos estacionarios los valores del vector de estados $\mathbf{x_e}$ y del vector de entradas $\mathbf{u_e}$ para los cuales el estado y, consecuentemente, la salida del sistema permanecen constantes,

\begin{align}
\mathbf{\dot{x}_e} \equiv 0 = \mathbf{f}(\mathbf{x_e},\mathbf{u_e})\\
\mathbf{y_e} = \mathbf{g}(\mathbf{x_e},\mathbf{u_e})
\end{align}
\end{definition}

Si el vector de estados no cambia, su derivada será cero. Por tanto, mientras no se altere el valor de la entrada, el sistema permanecerá en el mismo estado y el valor del vector de salidas permanecerá también constante

Podemos, a partir de esta definición, obtener algunas propiedades importantes de los puntos de equilibrio:
\begin{enumerate}
\item Una vez que un sistema alcanza un punto de equilibrio, permanece en él indefinidamente. (Todas las derivadas temporales de las componentes del vector de estado son cero.
\item Desde el punto de vista del control de sistemas, los puntos de equilibrio juegan un papel importante ya que representan condiciones de operación constante.
\item Un sistema dinámico puede tener uno o más puntos de equilibrio, o no tener ninguno. 
\end{enumerate}

\paragraph{Sistemas autónomos.} Para simplificar el estudio de la estabilidad, podemos empezar por considerar el casos de sistemas que tienen entrada nula $\mathbf{u} = \mathbf{0}, \ \forall t$. en los que la entrada es una función directa de los estados del sistema. Hablaremos entonces de un \emph{sistema autónomo}; la salida evoluciona a partir de un estado inicial $\mathbf{x_0}\equiv \mathbf{x(t_0)}$,

\begin{align}
\mathbf{\dot{x}} = \mathbf{f}(\mathbf{x})\\
\mathbf{y} = \mathbf{g}(\mathbf{x})
\end{align}

\paragraph{Sistemas realimentados.} Del mismo modo, podemos considerar sistemas en los que la entrada es una función directa del valor de los estados; $\textbf{u}= \textbf{c(x)}$. Hablaremos entonces de un \emph{sistema realimentado}. A efectos de análisis de la estabilidad del sistema, no hay diferencia entre un sistema realimentado y un sistema autónomo,
\begin{align}
\mathbf{\dot{x}} = \mathbf{f}(\mathbf{x},\mathbf{c(x)}) = \bar{f}(\mathbf{x})\\
\mathbf{y} = \mathbf{g}(\mathbf{x},\mathbf{c(x)}) = \bar{\mathbf{g}}(\mathbf{x})
\end{align}
El nombre de sistema realimentado, proviene de considerar que se están \emph{realimentando} los estados en la entrada del sistema. El concepto de realimentación constituye uno de los pilares de los sistemas de control. La figura \ref{fig:rea}, muestra esquemáticamente este concepto.

\begin{figure}
\centering
\begin{tikzpicture}
\draw(0,0) node(p)[rectangle, minimum height=10mm, minimum width=10mm,align=center, very thick, draw = blue!50!black!50]{$\mathbf{f}(\mathbf{x}(t),\mathbf{u}(t))$}

(0,-1.5) node(c)[rectangle, minimum height=10mm, minimum width=10mm,align=center ,very thick,draw = red!60!black!40]{$\mathbf{k}(\mathbf{x}(t))$};

\draw[line width = 1pt](p)--(2,0)node[midway,above]{$\mathbf{x}(t)$};
\draw[line width = 1pt](2,0)--(2,-1.5);
\draw[line width = 1pt, -latex](2,-1.5)--(c);
\draw[line width = 1pt](c)--(-2,-1.5);
\draw[line width = 1pt](-2,-1.5)--(-2,0);
\draw[line width = 1pt, -latex](-2,0)--(p)node[midway,above]{$\mathbf{u}(t)$};

\end{tikzpicture}

\caption{Esquema general de un sistema realimentado}\label{fig:rea}
\end{figure}

Tanto para un sistema autónomo como para uno realimentado los puntos de equilibrio deberán satisfacer la condición de que las derivadas temporales de los estados sean todas nulas,
\begin{equation}
\mathbf{0} = \mathbf{f(x_e})
\end{equation}

\section{Estabilidad de Lyapunov}
La teoría de la estabilidad constituye un aspecto importante dentro del estudio de los sistemas dinámicos. Puede abordarse desde diversos puntos de vista. En esta sección vamos a centrarnos en el estudio de la estabilidad de los puntos de equilibrio. 

Habitualmente, cuando se trata de la estabilidad de los puntos de equilibrio se suele hablar de estabilidad en el \emph{sentido} de Lyapunov. Aleksandr Mikhailovich Lyapunov un matématico ruso, de finales del siglo XIX, que estableció los fundamentos de la teoría de la estabilidad que ahora lleva su nombre.

Un punto de equilibrio de un sistema dinámico es \emph{estable} si toda  solución que comienzan cerca del punto de equilibrio permanece cercana al punto de equilibrio; en otro caso, el punto de equilibrio es inestable. Es \emph{asintóticamente estable} si toda solución que empieza próxima al punto de equilibrio no solo permanecen cerca del punto de equilibrio sino que tienden a él cuando el tiempo tiende a infinito.

\subsection{Estabilidad de Lyapunov para sistemas autónomos}
Consideremos un sistema autónomo genérico,
\begin{equation}\label{eq: aut}
\dot x = f(x),
\end{equation}
donde $f(x): D\rightarrow \mathbb{R}^n$ es una función (localmente) lipschitziana desde un domino $D \subset \mathbb{R}^n$ a $\mathbb{R}^n$. Supongamos además que $\overline x \in D$ es un punto de equilibrio; es decir, $f(\overline x) = 0$. Lo que buscamos es un método para caracterizar el tipo de estabilidad de $\overline x$. En lo que sigue, y sin perdida de generalidad, consideramos que el punto de equilibrio es es origen de coordenadas $\overline x = 0$ \footnote{Siempre es posible hacer un cambio de coordenadas de modo que si $\overline x \neq 0$, $z = x-\overline x$; $\dot z = \dot x = f(x+\overline x) := g(z)$, donde $g(0)=0$.}.

\begin{definition}
El punto de equilibrio x = 0 del sistema \ref{eq: aut} es:
\begin{itemize}
\item estable $\forall \epsilon > 0$ que cumpla:
\begin{equation*}
 \exists \delta =\delta(\epsilon)>0: \| x(0) \| < \delta \Rightarrow \| x(t)\| < \epsilon \ \forall t \ge 0
\end{equation*}
 
 \item Es inestable si no es estable.
 \item Es asintóticamente estable si es estable y además se puede elegir $delta$ de modo que,
\begin{equation*}
\| x(0)\| < \delta \Leftarrow \lim_{t \to \infty} x(t) = 0
\end{equation*}
\end{itemize}
\end{definition}

En 1892 Lyaponov propuso un método para analizar la estabilidad de un punto de equilibrio, basado en el análisis  una función $V:D\to \mathbb{R}$, continua y diferenciable en un entorno $D \subset \mathbb{R}^n$. En concreto, el método analiza la variación de la función $V$ a lo largo de las \emph{trayectorias} del sistema cuya estabilidad se quiere analizar, Podemos representar dicha variación empleando la derivada temporal de la función $V$
\begin{equation}
\dot V(x) = \sum_{i=0}^{n}\frac{\partial V}{\partial x_i} \dot x_i = \sum_{i=0}^{n}\frac{\partial V}{\partial x_i} f(x) = \frac{\partial V}{\partial x}f(x)
\end{equation}

Es interesante hacer notar que, lógicamente, $dot V$ depende del sistema que se desea analizar y que Si $\phi(t;x)$ es una solución del sistema \ref{eq: aut} que tiene como condición inicial estado $x$, para $t=0$ Entonces,
\begin{equation}
\dot V(x) =\left. \frac{d}{dt}V(\phi (t;x))\right|_{t=0}
\end{equation}
\section{Principio de invarianza de LaSalle}


\chapter{Control methodologies for state-space systems}

\section{Trajectory tracking}
Consider the following second-order kinematic system
\begin{equation}
	\ddot p(t) = u,
	\label{eq: pdyn}
\end{equation}
where $p,u\in\mathbb{R}^n$ are the \emph{positions} and the \emph{acceleration} inputs respectively in a $l\in\mathbb{N}$-dimensional space. Let us focus on $l=2$, then we will have a linear system with
\begin{equation}
	A = \begin{bmatrix}0 & 1 & 0 & 0 \\ 0 & 0 & 0 & 0 \\
		0 & 0 & 0 & 1 \\ 0 & 0 & 0 & 0 \end{bmatrix}, \quad B = \begin{bmatrix}0 & 0  \\ 1 & 0  \\ 0 & 0 \\ 0 & 1\end{bmatrix},
\end{equation}
and for now we assume that we can measure all the states, i.e., $y = x$, so $C = I_4$.

We define as trajectory the desired signal to be tracked, i.e.,
\begin{equation}
	p^*(t) = f(t), \quad \dot p^*(t) = f'(t),
\end{equation}
where $f(t) \in C^2$ function.
Let us construct the following error signals for the states
\begin{equation}
	e_1(t) = p(t) - p^*(t), \quad e_2(t) = \dot p(t) - \dot p^*(t).
\end{equation}
Now consider the following Lyapunov candidate function
\begin{equation}
	V(e(t)) = \frac{1}{2}||e(t)||^2,
	\label{eq: Ve}
\end{equation}
	where $e(t)\in\mathbb{R}^4$ is the stacked error vector with components $e_1, e_2$. The time derivative of ($\ref{eq: Ve}$) satisfies
\begin{align}
	\frac{\mathrm{d}V}{\mathrm{dt}} &= e^T\dot e = e_1^T\dot e_1 + e_2^T\dot e_2 =  e_1^T(\dot p - f'(t)) + e_2^T(u - f''(t)) \nonumber \\
	&= e_1^Te_2 + e_2^T(u - f''(t))
\end{align}
	if one chooses 
	\begin{equation}
	u = f''(t) - e_1 - e_2 = f''(t) - p(t) + f(t) -\dot p(t) + f'(t),
	\label{eq: ue}
	\end{equation}
	leads to
	\begin{equation}
\frac{\mathrm{d}V}{\mathrm{dt}} = -||e_2||^2 \leq 0.
	\end{equation}
	In fact, the time derivative of $V(t)$ is only zero when $e_2 = 0$. Therefore, by invoking the LaSalle's invariance principle, we must check that the largest invariant set for the dynamics of the (autonomous) error system (when $u$ equals (\ref{eq: ue})) when $e_2 = 0$, i.e., 
	\begin{equation}
	\begin{cases}
	\dot e_1 &= e_2 \\
	\dot e_2 &= -e_1 -e_2
	\end{cases},
	\end{equation}
	and we can see that when $e_2 = 0$, the largest invariant set is $e_1 = e_2 = 0$. Consequently, we can conclude that $e(t) \to 0$ as $t\to\infty$.

	\section{Design of limit cycles for planar systems}
	Now we want to attract the position of system (\ref{eq: pdyn}) to the \emph{curve} $\phi(p_x, p_y) = 0$, which is an implicit equation where the position $p$ is at the level set zero. For example, for a circle of radius $r\in\mathbb{R}_+$ we might consider
	\begin{equation}
	\phi(p) := p_x^2 + p_y^2 - r^2 = 0.
	\end{equation}
	Note that we can consider $\phi(p)$ as an error signal, i.e., $e(t) := \phi(p(t))$. Now, consider the following Lyapunov candidate
\begin{equation}
	V(e(t)) = \frac{1}{2}e(t)^2 + \frac{1}{2}||\dot p(t) - \dot p^*(t)||^2,
	\label{eq: Ve2}
\end{equation}
	whose time derivative satisfies (for now consider $\dot p^*(t) = 0$)\footnote{Recall that the gradient is written as a row vector.}
	\begin{equation}
		\frac{\mathrm{d}V}{\mathrm{dt}} = e\dot e + \dot p^T\ddot p = e \nabla\phi(p) \dot p + \dot p^T u,
	\end{equation}
	therefore if one chooses $u = -e \nabla\phi(p)^T -\dot p$, then we have that
	\begin{equation}
\frac{\mathrm{d}V}{\mathrm{dt}} = -||\dot p||^2 \leq 0,
	\end{equation}
and by LaSalle's invariance principle we can conclude that $e(t) \to 0$ as $t\to\infty$.

What would happen if $\dot p^*(t) = \begin{bmatrix}0 & 1 \\ -1 & 0\end{bmatrix}\nabla\phi(p)^T$?


\chapter{Sistemas lineales}

\section{Mapas lineales}

En este capítulo nos vamos a centrar en una clase de sistema llamado \emph{sistema linea en el espacio de estados}. Primero, necesitamos la noción de que es un \emph{mapa lineal}.

\begin{definition} Considera un mapeado $H: V \to W$. Si $H$ preserva la operación suma y la multiplicación por un escalar, i.e.,
\begin{align}
	H(v_1+v_2) &= H(v_1) + H(v_2), \quad v_1, v_2\in\mathbb{V} \nonumber \\
	H(\alpha v_1) &= \alpha H(v_1), \quad \alpha\in\mathbb{K} \nonumber,
\end{align}
	entonces $H$ es un \emph{mapa lineal}.
\end{definition}

\subsection{Ejercicio: Comprueba si los siguientes mapas son lineales}

\begin{enumerate}
	\item $H_1(v) := Av, A\in\mathbb{R}^{n\times n}, \quad v\in\mathbb{R}^n$
	\item $H_2(v) := \frac{\mathrm{d}}{\mathrm{dt}}(v(t)), \quad v\in\mathcal{C}^1$
	\item $H_3(v) := \int_0^T v(t) dt, \quad v\in\mathcal{C}^1, T\in\mathbb{R}_{\geq 0}$
	\item $H_4(v) := D(v) := v(t - T), \quad v\in\mathcal{C}^1, T\in\mathbb{R}_{\geq 0}$
	\item $H_5(v) := Av + b, \quad A\in\mathbb{R}^{n\times n}, v,b\in\mathbb{R}^n$
\end{enumerate}

\section{Sistemas continuos y lineales en el espacio de estados}

El siguiente sistema define un sistema continuo y lineal en el espacio de estados.

\begin{equation}
	\Sigma := \begin{cases}
	\dot x(t) &= A(t)x(t) + B(t)u(t), \quad x\in\mathbb{R}^n, u\in\mathbb{R}^k \\
	y(t) &= C(t)x(t) + D(t)u(t), \quad y\in\mathbb{R}^m
	\end{cases}
	\label{eq: linsys}
\end{equation}

\subsection{Ejercicio: Escribe un sistema continuo y lineal en el espacio de estados como un diagrama de bloques entrada/salida y comprueba que es un mapa lineal.}

\subsection{Ejercicio: Interconecta sistemas continuos y lineales en el espacio de estados y comprueba que el sistema resultante es otro sistema continuo y lineal en el espacio de estados.}

Reescribe como un único sistema lineal\footnote{Por abreviar, cuando no exista ambiguedad, llamaremos sistema lineal al sistema continuo y lineal en el espacio de estados} como en (\ref{eq: linsys}):

\begin{enumerate}
	\item La conexión en serie (o en cascada) de dos sistemas lineales, i.e., $y_1(t) = u_2(t)$.
	\item La conexión en paralelo de dos sistemas lineales, i.e., $y(t) = y_1(t) + y_2(t)$.
	\item La conexión realimentada, i.e., $u_1(t) = u(t) - y(t)$, asumiendo que $u, y, \in\mathbb{R}^k$.
\end{enumerate}

\begin{figure}
\centering
\begin{tikzpicture}[auto, node distance=3.5cm, >=latex']
	\node [input, name=input] {};
	\node [block, right of=input] (system) {$\Sigma_1$};
	\node [output, right of=system] (output) {};
	\draw [draw,->] (input) -- node {$u(t) = u_1(t)$} (system);
	\draw [->] (system) -- node [name=y] {$y_1(t)$}(output);
	\node [block, right of=output] (system2) {$\Sigma_2$};
	\node [output, right of=system2] (output2) {};
	\draw [draw,->] (output) -- node {$u_2(t)$} (system2);
	\draw [->] (system2) -- node [name=y] {$y_2(t) = y(t)$}(output2);
\end{tikzpicture}
	\caption{Conexión en serie de dos sistemas lineales y continuos en el espacio de estados.}
	\label{fig: series}
\end{figure}

\section{Solución a sistemas continuos lineales en el espacio de estados}
La solución a una ecuación diferencial ordinaria viene dada por la suma de dos soluciones: la solución a la parte homogénea, y la solución a la parte no homogénea.

\begin{equation}
	\dot x(t) = \underbrace{A(t)x(t)}_{\text{homogénea}} + \underbrace{B(t)u(t)}_{\text{no homogénea}}
	\label{eq: xdyn}
\end{equation}

\begin{theorem}{Serie de Peano-Barker.}
La solución única al sistema homogéneo $\dot x = Ax$ viene dada por
	\begin{equation}
		x(t) = \Phi(t,t_0)x(t_0), \quad x(t_0)\in\mathbb{R}^n, t\geq 0,
	\end{equation}
donde
	\begin{align}
		\Phi(t,t_0) := I + \int_{t_0}^t A(s_1)ds_1 + \int_{t_0}^t A(s_1) \int_{t_0}^{s_1} A(s_2)ds_2ds_1 \nonumber \\ + \int_{t_0}^t A(s_1) \int_{t_0}^{s_1} A(s_2)\int_{t_0}^{s_2} A(s_3) ds_3ds_2ds_1 + \dots . \label{eq: ser}
	\end{align}
\end{theorem}
Esbozo de la prueba: \\
Primero calculamos la siguiente derivada
	\begin{align}
		\frac{d}{dt}\Phi(t,t_0) &= A(t) + A(t)\int_{t_0}^{t}A(s_2)ds_2 \nonumber \\ &+ A(t)\int_{t_0}^t A(s_2) \int_{t_0}^{s_2} A(s_3)ds_3ds_2 + \dots \nonumber \\
		&= A(t) \Phi(t,t_0).
	\end{align}
	Afirmamos que la solución a la parte homogénea de (\ref{eq: xdyn}) es $x(t) = \Phi(t,t_0)x_0$ cuya derivada con respecto al tiempo es
\begin{align}
	\frac{d}{dt} x &= \frac{d}{dt}\Phi(t,t_0)x_0 \nonumber \\
	&= A(t) \Phi(t,t_0) x_0 \nonumber \\
	&= A(t)x(t),
\end{align}
lo cual prueba la identidad $\dot x = A(t)x(t)$ dado que $x(t) = \Phi(t,t_0)x_0$. Para terminar la prueba, necesitaríamos probar que la serie (\ref{eq: ser}) converge para todo $t\geq t_0$.

La matriz $\Phi(t,t_0)$ es llamada \textbf{\emph{matriz de transición de estados}}. Dada una condición inicial $x_0$, podemos predecir $x(t)$ en (\ref{eq: xdyn}) iterando $\Phi(t,t_0)$ en el caso de que no existiera ninguna interacción con el sistema, i.e., $u(t) = 0, t\geq t_0$.

\subsection{Ejercicio}
Comprobar que
\begin{align}
	x(t) &= \Phi(t,t_0)x_0 + \int_{t_0}^t \Phi(t,\tau)B(\tau)u(\tau)d\tau  \label{eq: solx} \\
	y(t) &= C(t)\phi(t,t_0)x_0 + \int_{t_0}^t C(t)\Phi(t,\tau)B(\tau)u(\tau)d\tau + D(t)u(t), \label{eq: soly}
\end{align}
son las soluciones a

\begin{align}
	\dot x(t) &= A(t)x(t) + B(t)u(t)  \nonumber \\
	\dot y(t) &= C(t)x(t) + D(t)u(t).  \nonumber
\end{align}

\section{Solución a sistemas invariantes en el tiempo, continuos y lineales en el espacio de estados}

Comunmente conocidos como sistemas \emph{lti} (linear time invariant), son los sistemas en los que nos centraremos principalmente en el resto del curso. La matriz $\Phi(t,t_0)$ puede ser hallada analíticamente cuando $A$ es una matriz de coeficientes constantes. Si $A$ es constante, entonces podemos sacarla de las integrales en (\ref{eq: ser}), quedando
\begin{align}
	\Phi(t,t_0) := I + A \int_{t_0}^t ds_1 + A^2 \int_{t_0}^t \int_{t_0}^{s_1} ds_2ds_1 \nonumber \\ + A^3 \int_{t_0}^t \int_{t_0}^{s_1} \int_{t_0}^{s_2} ds_3ds_2ds_1 + \dots \label{eq: phi},
\end{align}
y observando que las siguientes integrales tienen solución analítica
\begin{align}
	\int_{t_0}^t ds_1 &= (t-t_0) \nonumber \\
	\int_{t_0}^t\int_{t_0}^{s_1} ds_2ds_1 &= \frac{(t-t_0)^2}{2} \nonumber \\
	\vdots \nonumber \\
	\int_{t_0}^t\int_{t_0}^{s_1} \cdots \int_{t_0}^{s_{k-2}}\int_{t_0}^{s_{k-1}}ds_k ds_{k-1} \cdots ds_2ds_1 &= \frac{(t-t_0)^k}{k!}, \nonumber
\end{align}
entonces tenemos que (\ref{eq: phi}) es calculada como
\begin{equation}
	\Phi(t,t_0) = \sum_{k=0}^{\infty} \frac{(t-t_0)^k}{k!}A^k,
\end{equation}
lo cual es familiar a la serie de Taylor de una función exponencial. Por ejemplo, para un escalar $x$, tenemos que $e^x := \sum_{k=0}^{\infty}\frac{1}{k!}x^k = 1 + x + \frac{x^2}{2} + \frac{x^3}{3!} + \dots $. De hecho, la definición de la \emph{exponencial de una matriz} es
\begin{equation}
	exp(A) = I + A + \frac{1}{2} A^2 + \frac{1}{3!} A^3 + \dots
\end{equation}
Fijemos $t_0 = 0$ por conveniencia, entonces
\begin{align}
	\Phi(t,0) &= I + tA + \frac{t^2}{2} A^2 + \frac{t^3}{3!} A^3 + \dots \nonumber \\
	&= exp(At),
\end{align}
por lo tanto, la solución a la parte homogénea (\ref{eq: xdyn}) teniendo $A$ con coeficientes constantes y fijando $t_0 = 0$ es
\begin{equation}
	x(t) = exp(At)x_0,\quad t\geq 0.
	\label{eq: xexp}
\end{equation}

Para continuar, necesitamos el siguiente resultado de álgebra lineal.
\begin{theorem}
\textbf{Forma de Jordan}. Para una matriz cuadrada $A\in\mathbb{C}^{n \times n}$, existe un cambio de base no singular $P\in\mathbb{C}^{n \times n}$ que transforma $A$ en
\begin{equation}
	J = PAP^{-1} = \begin{bmatrix}
		J_1 & 0 & 0 & \dots & 0 \\
		0 & J_2 & 0 & \dots & 0 \\
		0 & 0 & J_3 & \dots & 0 \\
		\vdots & \vdots & \vdots & \cdots & \vdots \\
		0 & 0 & 0 & \cdots & J_l
	\end{bmatrix},
\end{equation}
donde $J_i$ es el bloque de Jordan con forma
	\begin{equation}
	J_i = \begin{bmatrix}
\lambda_i & 1 & 0 & \dots & 0 \\
		0 & \lambda_i & 1 & \dots & 0 \\
		0 & 0 & \lambda_i & \dots & 0 \\
		\vdots & \vdots & \vdots & \cdots & \vdots \\
		0 & 0 & 0 & \cdots & \lambda_i
	\end{bmatrix}_{n_i\times n_i},
	\end{equation}
	en donde cada $\lambda_i$ es un autovalor de $A$, y el número $l$ de bloques de Jordan es igual al número total de autovectores independientes de $A$. La matrix $J$ es única (descontando reordenación de filas/columnas) y es llamada la \textbf{forma normal de Jordan} de $A$.
\end{theorem}

Partiendo de la observación que $A = P^{-1}JP$ también, entonces es fácil probar que 
\begin{equation}
	A^k = P^{-1} J^k P,
\end{equation}
de tal manera que podamos calcular que
\begin{align}
	exp(At) &= P^{-1}\left(\sum_{k=1}^\infty \frac{t^k}{k!} \begin{bmatrix}J_1^k & 0 & \cdots & 0 \\ 0 & J_2^k & \cdots & 0 \\ \vdots & \vdots & \cdots & \vdots \\ 0 & 0 & \cdots & J_l^k \end{bmatrix} \right) P \nonumber \\
		&= P^{-1} \begin{bmatrix}exp(J_1t) & 0 & \cdots & 0 \\ 0 & exp(J_2t) & \cdots & 0 \\ \vdots & \vdots & \cdots & \vdots \\ 0 & 0 & \cdots & exp(J_lt) \end{bmatrix} P
\end{align}

Observa que si $J$ es simplemente una matriz diagonal con los autovalores de $A$, i.e., $J_l = \lambda_l \in \mathbb{C}$, entonces $exp(J_lt) = e^{\lambda_lt} \in\mathbb{C}$ es un cálculo trivial.

Now, let us check the consequencues on the following two conditions
Ahora, veamos las consecuencias de las siguientes dos suposiciones
\begin{enumerate}
	\item $J$ es diagonal.
	\item Todos los autovalores de $A$ tienen parte real negativa.
\end{enumerate}

Sabiendo que $\lim_{t\to\infty} e^{\lambda t} \to 0$ si $\lambda \in \mathbb{R}_{<0}$, entonces tenemos que $exp(At) \to 0$ según $t\to\infty$ si las dos previas suposiciones se dan. Si echamos un vistazo a (\ref{eq: xexp}), podemos concluir que 
\begin{equation}
	\lim_{t\to\infty} x(t) \to 0,
	\label{eq: xlim}
\end{equation}
por tanto, podemos predecir la evolución de $x(t)$ con sólamente mirar los autovalores de $A$. Si $J$ no es diagonal, podremos concluir más resultados. Lo veremos en la sección siguiente a la linearización de sistemas en el espacio de estados.


\section{Linearización de sistemas en el espacio de estados}
Desafortunadamente, es realmente dificil (cuando no imposible) calcular una solución analítica para $x(t)$ e $y(t)$ para un sistema arbitrario $\Sigma$ como en (\ref{eq: sigma}). No obstante, hemos visto que sí se puede calcular una solución analítica para $x(t)$ e $y(t)$ cuando $\Sigma$ es un sistema invariante en el tiempo, continuo y lineal en el espacio de estados.

Será de gran utilidad encontrar una relación entre ambos sistemas.

Si $f(x,t)$ y $g(x,t)$ son reales analíticas en un entorno a un punto específico $(x^*,u^*)$, entonces podemos trabajar con aproximaciones de Taylor de $f(x,t)$ y $g(x,t)$ en ese mismo entorno. Cuando nos quedamos en orden uno en la aproximación es lo que se conoce como \emph{linearización}.
\begin{equation}
	\Sigma := \left.\begin{cases}
	\dot x(t) =& f(x(t),u(t)) \\ y(t) =& g(x(t),u(t))
	\end{cases}\right|_{x\approx x^*, u\approx u^*} \approx
	\begin{cases}
		x(t) &= x^* + \delta x(t) \\
		u(t) &= u^* + \delta u(t) \\
	\delta \dot x(t) &= A(t)\delta x(t) + B(t)\delta u(t) \\
	\delta y(t) &= C(t)\delta x(t) + D(t)\delta u(t)
	\end{cases}, \nonumber
\end{equation}
donde
\begin{align}
	A(t) &= \begin{bmatrix}
		\frac{\partial f_1}{\partial x_1} & \dots & \frac{\partial f_1}{\partial x_n} \\
		\vdots & \vdots & \vdots \\
		\frac{\partial f_n}{\partial x_1} & \dots & \frac{\partial f_n}{\partial x_n}
	\end{bmatrix}_{|_{x=x^*, u=u^*}} \quad
	&B(t) = \begin{bmatrix}
		\frac{\partial f_1}{\partial u_1} & \dots & \frac{\partial f_1}{\partial u_k} \\
		\vdots & \vdots & \vdots \\
		\frac{\partial f_k}{\partial u_1} & \dots & \frac{\partial f_k}{\partial u_k}
	\end{bmatrix}_{|_{x=x^*, u=u^*}} \nonumber \\
	C(t) &= \begin{bmatrix}
		\frac{\partial g_1}{\partial x_1} & \dots & \frac{\partial g_1}{\partial x_n} \\
		\vdots & \vdots & \vdots \\
		\frac{\partial g_m}{\partial x_1} & \dots & \frac{\partial g_m}{\partial x_n}
	\end{bmatrix}_{|_{x=x^*, u=u^*}} \quad
	&D(t) = \begin{bmatrix}
		\frac{\partial g_1}{\partial u_1} & \dots & \frac{\partial g_1}{\partial u_k} \\
		\vdots & \vdots & \vdots \\
		\frac{\partial g_m}{\partial u_1} & \dots & \frac{\partial g_m}{\partial u_k}
	\end{bmatrix}_{|_{x=x^*, u=u^*}}. \nonumber
\end{align}
Informalmente, estamos calculando la sensibilidad (hasta primer orden) de $f$ y $g$ cuando hacemos una variación pequeña de $x$ y $u$ alrededor de $(x^*,u^*)$. Como de pequeña ha de ser esa variación depende del sistema $\Sigma$. En particular, cuando diseñemos controladores basados en linearizar alrededor de un punto, daremos cotas para $\delta x$ y $\delta u$ de tal manera que el controlador pueda garantizar estabilidad.

\subsection{Ejercicio. Linearización del péndulo invertido}
Más adelante, veremos que podemos diseñar una entrada de control $u(t)$, i.e., una señal que ha de seguir el torque $T$ en (\ref{eq: f}) de tal manera que $\theta$ y $\dot\theta$ converjan a unos valores constantes o trayectorias deseadas.

Por ejemplo, vamos a fijar un punto constante de interés $x^* = \begin{bmatrix}\theta^* \\ 0\end{bmatrix}$, por lo que la velocidad angular se marca a cero. Esta situación corresponde a una situación de equilibrio para el ángulo $\theta$. Para hallar el $u^*(t)$ en (\ref{eq: f}) necesario para tal equilibrio necesitamos que $\frac{\mathrm{d}}{\mathrm{dt}}\left(\begin{bmatrix}\theta \\ \dot\theta \end{bmatrix}\right) = \begin{bmatrix}0 \\ 0 \end{bmatrix}$. Una inspección a la dinámica (\ref{eq: dyn}) nos responde que
\begin{equation}
	u^* = T^* = -\frac{g}{l}\sin\theta^*,
\end{equation}
por ejemplo, para una posición totalmente vertical correspondiente a $\theta^* = 0$ tenemos que $T^*=0$, i.e., $x^* = \begin{bmatrix}0\\0\end{bmatrix}$ y $u^* = 0$.

El cáclulo de las matrices $A,B,C,$ y $D$ son los Jacobianos de $(\ref{eq: f})$ y $(\ref{eq: g})$, i.e.,

\begin{align}
\frac{\partial f_1}{\partial x_1} &= 0 \nonumber \\
\frac{\partial f_1}{\partial x_2} &= 1 \nonumber \\
\frac{\partial f_2}{\partial x_1} &= \frac{g}{l}\cos\theta \nonumber \\
\frac{\partial f_2}{\partial x_2} &= -\frac{b}{ml^2} \nonumber \\
\frac{\partial f_1}{\partial u_1} &= 0 \nonumber \\
\frac{\partial f_2}{\partial u_1} &= 1 \nonumber \\
\frac{\partial g_1}{\partial x_1} &= 1 \nonumber \\
\frac{\partial g_1}{\partial x_2} &= 0 \nonumber \\
\frac{\partial g_1}{\partial u_1} &= 0, \nonumber
\end{align}
por lo que podemos llegar a
\begin{align}
	\frac{\mathrm{d}}{\mathrm{dt}}\left(\begin{bmatrix}\delta\theta \\ \dot\delta\theta \end{bmatrix}\right) &= \begin{bmatrix}0 & 1 \\ \frac{g}{l}\cos\theta & -\frac{b}{ml^2} \end{bmatrix}_{|_{\theta=\theta^*}} \begin{bmatrix}\delta\theta \\ \dot\delta\theta \end{bmatrix} + \begin{bmatrix}0 \\ 1 \end{bmatrix} \delta T \nonumber \\
		\delta y &= \begin{bmatrix}1 & 0\end{bmatrix}\begin{bmatrix}\delta\theta \\ \dot\delta\theta \end{bmatrix} + 0 \, \delta T,
\end{align}
para modelar una aproximación a la dinámica de $x(t)$ y la salida $y(t)$ alrededor de los puntos $x^*$ y $u^*$.

\printindex
\end{document} 
