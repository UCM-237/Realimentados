\documentclass[10pt,a4paper]{article}
\usepackage[utf8]{inputenc}
\usepackage[spanish]{babel}
\usepackage{graphicx}
\usepackage{subcaption}
\usepackage{fancyhdr}
\usepackage{amsmath}
\usepackage{amsfonts}
\usepackage{amssymb}
\usepackage{eurosym}
\usepackage{tikz}
\usetikzlibrary {arrows.meta}
%\usepackage[includehead=true,includefoot=true,portrait,top = 1cm]{geometry}
\usepackage[left=4cm,right=4cm,top=2cm,bottom=5cm]{geometry}
\graphicspath{{./FIGURAS/}}

%\fancyhf{}
\headsep = 2cm
\lhead{Nombre:}
%\lhead{{\huge\textbf{B}} Nombre:}
\chead{}
\rhead{\includegraphics[scale=0.05]{gallinita.pdf}}
\lfoot{}
\cfoot{}
\rfoot{\today}
\newcommand{\valve}[3]{%
    \draw (#1,#2-.5) -- (#1,#2+.5) -- (#1+2,#2-.5) -- (#1+2,#2+0.5) -- cycle;
    \draw (#1+1,#2) -- (#1+1,#2+0.5);
    \draw (#1+0.75,#2+0.5) rectangle (#1+1.25,#2+0.7);
    \node at (#1+1,#2-1) {\bf #3};
}

\newcommand{\process}[3]{%
    \draw [rounded corners=2.5mm,fill=white, draw = black] (#1-1,#2-0.5) rectangle (#1+1,#2+2);
    \node at (#1,#2) {\bf #3};
}

\newcommand{\pump}[5]{%
        \filldraw [color=white,draw=black](#1,#2) circle (0.6);
        \draw[arrows = {-Latex[width'=5pt .5, length=10pt]}](#1,#2)--(#1+0.6,#2);
        \node at (#1+#3,#2+#4) {\bf #5};
}

\def\labelitemi{$\square$}
\begin{document}
\thispagestyle{fancy}

\section*{Exámen de Sistemas Dinámicos y Realimentación\footnote{Las respuestas pueden subirse al Campus Virtual en un \emph{live script}, pdf o similar, incluyendo los códigos utilizados. Si alguien prefiere entregar las explicaciones y deducciones en papel, también puede hacerlo.}.}

\begin{enumerate}
\item Dado el sistema lineal
\begin{align*}
\dot{x} &= Ax+Bu, \\
y &= Cx, \\
x \in \mathbb{R}^3; y,u \in \mathbb{R}\\
\text{con,}\\
A &=\begin{bmatrix}
-2& 1& 0& 1\\
 1& -3 &1 &1\\
 0& 1 &-1 &0\\
 0& 1 &1 &-2
\end{bmatrix}, B=\begin{bmatrix}
1\\
0\\
0\\
1
\end{bmatrix},
C = \begin{bmatrix}
 1&1&0&0
\end{bmatrix}
\end{align*}
\begin{enumerate}
\item (1 punto) Dicutir analíticamente la estabilidad del sistema

\item \label{b} (1 punto) Comprueba que el sistema es controlable y diseña un sistema de estabilización por realimentación de estados de modo que los autovalores estén situados en $[-1, -1, -1, -1]$. Compara gráficamente la respuesta del sistema en lazo abierto con la del sistema realimentado, suponiendo una entrada nula y condiciones iniciales $x_1=-1$, $x_2=-2$, $x_3=2$, $x_4=1$. 

\item \label{c} (1.5 puntos) Comprueba que no es posible emplear control integral para situar a la vez las variables $x_1$ y $x_4$ en posiciones arbitrarias. Demuestra que sí es posible situar arbitrariamente la variable $x_3$. Añade al sistema de control diseñado en el apartado (\ref{b}) control integral, coloca el autovalor de la acción integral en $-1$ y emplealo para desplazar la variable $x_3$ una distancia $x_3 = -4$ de su posición de equilibrio. Muestra gráficamente los resultados


\item (1.5 puntos) Diseña un estimador de modo que sus autovalores estén desplazados cuatro unidades hacia a la izquierda respecto a los polos del sistema controlado diseñado en el apartado (\ref{b}). Emplea el estimador para realimentar y controlar el sistema. Repite el calculo realizado en el apartado (\ref{c}) pero llevando ahora $x_3=4$. Considera nulas las condiciones iniciales del estimador $\hat{x}_i=0$.

\end{enumerate}


\item Dado el sistema,
\begin{equation}\label{eq1}
\begin{split}
\dot x_1 &= x_2\\
 \dot x_2 &= -\sin(x_1) - 2x_2
\end{split}
\end{equation}
$x_1,x_2 \in \mathbb{R}$˘
\begin{enumerate}
\item (1 punto) Obtener los puntos de equilibrio del sistema. Linealizar en torno a ellos y discutir su estabilidad.
\item \label{apb} (2 puntos)  Emplea el criterio de estabilidad de Lyapunov para  estudiar la estabilidad del sistema no lineal. Puedes usar para ello la siguiente función candidata:
\begin{equation*}
V = 1-\cos(x_1)+ \frac{x_2^2}{2}
\end{equation*}

\item (2 puntos) Resuelve numéricamente la ecuación (\ref{eq1}), Para distintos valores de las condiciones iniciales, y obtén un diagrama de fases del sistema. Comprueba que los resultados son coherentes con el análisis realizado en los apartados anteriores.
\end{enumerate}
\end{enumerate}
\end{document}
